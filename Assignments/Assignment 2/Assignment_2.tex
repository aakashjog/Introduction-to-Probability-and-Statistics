\documentclass[fleqn, a4paper, 11pt, oneside]{amsart}
\usepackage{exsheets}
\usepackage{amsmath, amssymb, amsthm} %standard AMS packages
\usepackage{marginnote} %marginnotes
\usepackage{gensymb} %miscellaneous symbols
\usepackage{commath} %differential symbols
\usepackage{xcolor} %colours
\usepackage{cancel} %cancelling terms
\usepackage[free-standing-units, space-before-unit]{siunitx} %formatting units
\usepackage{tikz, pgfplots} %diagrams
	\usetikzlibrary{calc, hobby, patterns, intersections, decorations.markings}
\usepackage{graphicx} %inserting graphics
\usepackage{hyperref} %hyperlinks
\usepackage{datetime} %date and time
\usepackage{enumerate,enumitem} %numbered lists
\usepackage{float} %inserting floats
\usepackage{circuitikz}[american voltages, american currents] %circuit diagrams
\usepackage{algpseudocode} %algorithms
\usepackage{algorithm} %algorithms
\usepackage{booktabs}

\newcommand\numberthis{\addtocounter{equation}{1}\tag{\theequation}} %adds numbers to specific equations in non-numbered list of equations

\theoremstyle{definition}
\newtheorem{example}{Example}
\newtheorem{definition}{Definition}

\theoremstyle{theorem}
\newtheorem{theorem}{Theorem}

\makeatletter
\@addtoreset{section}{part} %resets section numbers in new part
\makeatother

\DeclareMathOperator{\comp}{\textsc{c}}

\DeclareMathOperator{\prob}{\mathrm{P}}

\newcommand*{\perm}[2]{{}^{#1}\mathrm{P}_{#2}}%
\newcommand*{\comb}[2]{{}^{#1}\mathrm{C}_{#2}}%

\SetupExSheets{solution/print = true}

%opening
\title
[
	Introduction to Probability and Statistics : Assignment 2
]
{
	Introduction to Probability and Statistics\\
	Assignment 2
}
\author
{
	Aakash Jog\\
	ID : 989323563
}
\date{\formatdate{16}{3}{2016}}

\begin{document}

\maketitle
%\setlength{\mathindent}{0pt}

\begin{question}
	If $4$ Americans, $3$ French people, and $3$ British people are to be seated in a row, how many seating arrangements are possible when people of the same nationality must sit next to each other?
	Assume that people of the same nationality are not identical.
\end{question}

\begin{solution}
	The number of ways to seat $4$ Americans together is $4!$.\\
	The number of ways to seat $3$ French people together is $3!$.\\
	The number of ways to seat $3$ British people together is $3!$.\\
	The number of ways to arrange these groups is $3!$.
	Therefore, the total number of seating arrangements is $4! 3! 3! 3!$.
\end{solution}

\begin{question}
	From $10$ married couples, we want to select a group of $6$ people that is not allowed to contain a married couple.
	\begin{enumerate}
		\item How many choices exist?
		\item How many choices exist if the group must also consist of $3$ men and $3$ women?
	\end{enumerate}
\end{question}

\begin{solution}
	\begin{enumerate}[leftmargin=*]
		\item
			The number of choices to select $6$ couples from $10$ is $\comb{10}{6}$.\\
			The number of choices to select $1$ person from each of these $6$ couples is $2^6$.\\
			Therefore, the total number of choices are $\comb{10}{6} 2^6$
		\item
			The number of choices to select $6$ couples from $10$ is $\comb{10}{6}$.\\
			The number of choices to select $3$ couples from which the men will be selected is $\comb{6}{3}$.\\
			Therefore, the total number of choices are $\comb{10}{6} \comb{6}{3}$.
	\end{enumerate}
\end{solution}

\begin{question}
	If there are no restrictions on where the digits and letters are placed, how many many $8$-character license plates, consisting of $5$ letters and $3$ digits are possible if no repetitions of letters or digits are allowed?
	What if the $3$ digits must be consecutive?
\end{question}

\begin{solution}
	The number of options to select the $5$ letters is $\comb{26}{5}$.\\
	The number of options to select the $3$ letters is $\comb{10}{3}$.\\
	If there are no restrictions on the positions, the number of arrangements of these $8$ characters is $8!$.
	Therefore, the total number of possible license plates is $\comb{26}{5} \comb{10}{3} 8!$.\\
	If the $3$ digits must be consecutive, the number of arrangements of the digits is $3!$.
	The number of ways of arrangements of the letters, and the consecutive three digits, is $6!$.
	Therefore, the total number of possible license plates is $\comb{26}{5} \comb{10}{3} 3! 6!$.\\
\end{solution}

\begin{question}
	An urn contains $20$ balls: $8$ black, $7$ red, and $5$ white.
	If $8$ balls are randomly selected, calculate the following probabilities.
	\begin{enumerate}
		\item $6$ black and $2$ white balls were selected.
		\item $3$ black, $2$ red, and $3$ white balls were selected.
		\item At least $1$ white ball was selected.
	\end{enumerate}
\end{question}

\begin{solution}
	\begin{enumerate}[leftmargin=*]
		\item
			The total number of ways to select $8$ balls is $\comb{20}{8}$.\\
			The possible number of ways of selecting $6$ black balls from $8$ is $\comb{8}{6}$.\\
			The possible number of ways of selecting $2$ white balls from $5$ is $\comb{8}{6}$.\\
			Therefore, the probability is $\frac{\comb{8}{6} \comb{5}{2}}{\comb{20}{8}}$.
		\item
			The total number of ways to select $8$ balls is $\comb{20}{8}$.\\
			The possible number of ways of selecting $3$ black balls from $8$ is $\comb{8}{3}$.\\
			The possible number of ways of selecting $2$ red balls from $7$ is $\comb{7}{2}$.\\
			The possible number of ways of selecting $3$ white balls from $5$ is $\comb{5}{3}$.\\
			Therefore, the probability is $\frac{\comb{8}{3} \comb{7}{2} \comb{5}{3}}{\comb{20}{8}}$.
		\item
			The total number of ways to select $8$ balls is $\comb{20}{8}$.\\
			The possible number of ways of selecting $8$ balls from $8$ black and $7$ red balls is $\comb{15}{8}$.\\
			Therefore, the probability of selecting $0$ white balls is $\frac{\comb{15}{8}}{\comb{20}{8}}$.\\
			Therefore, the probability of selecting at least $1$ white ball is $1 - \frac{\comb{15}{8}}{\comb{20}{8}}$.
	\end{enumerate}
\end{solution}

\begin{question}
	In a state lottery, a player must choose $8$ of the numbers from $1$ to $40$.
	The lottery commission then performs an experiment that selects $8$ of these $40$ numbers.
	Assuming that the choice of the lottery commission is equally likely to be any of the $\binom{40}{8}$ combinations, find the probabilities that a player has
	\begin{enumerate}
		\item all $8$ of the numbers selected by the lottery commission.
		\item $7$ of the numbers selected by the lottery commission.
		\item at least $6$ of the numbers selected by the lottery commission.
	\end{enumerate}
\end{question}

\begin{solution}
	\begin{enumerate}[leftmargin=*]
		\item
			The number of combinations with all of the $8$ selected numbers is $1$.\\
			Therefore the probability is $\frac{1}{\comb{40}{8}}$.
		\item
			The number of ways to choose $1$ of the non-selected numbers is $\frac{32}{1}$.\\
			The number of ways to choose $7$ of the selected numbers is $\comb{8}{7}$.\\
			Therefore, the number of ways to choose $7$ of the selected numbers is $\comb{32}{1} \comb{8}{7}$.\\
			Therefore, the probability is $\frac{\comb{32}{1} \comb{8}{7}}{\comb{40}{8}}$.
		\item
			The number of ways to choose $2$ of the non-selected numbers is $\frac{32}{2}$.\\
			The number of ways to choose $6$ of the selected numbers is $\comb{8}{6}$.\\
			Therefore, the number of ways to choose $6$ of the selected numbers is $\comb{32}{2} \comb{8}{6}$.\\
			The number of ways to choose $7$ of the selected numbers is $\comb{32}{1} \comb{8}{7}$.\\
			The number of ways to choose $8$ of the selected numbers is $1$.\\
			Therefore the number of ways to choose at least $6$ of the selected numbers is $\comb{32}{2} \comb{8}{6} + \comb{32}{1} \comb{8}{7} + 1$.\\
			Therefore, the probability is $\frac{\comb{32}{2} \comb{8}{6} + \comb{32}{1} \comb{8}{7} + 1}{\comb{40}{8}}$.
	\end{enumerate}
\end{solution}

\begin{question}
	$10$ married couples randomly select their seats around a round table.
	Find the following probabilities.
	\begin{enumerate}
		\item All couples sit next to each other.
		\item Only $9$ couples sit next to each other.
	\end{enumerate}
\end{question}

\begin{solution}
	\begin{enumerate}[leftmargin=*]
		\item
			The total possible number of arrangements is $19!$.\\
			The number of ways to arrange each couple amongst themselves is $2!$.
			Therefore, the number of ways to arrange all $10$ couples, amongst themselves, is $2^{10}$.\\
			The number of ways to arrange all couples, grouped together, is $9!$.\\
			Therefore, the total number of arrangements is $2^{10} 9!$.\\
			Therefore, the probability is $\frac{2^{10} 9!}{19!}$.
		\item
			The total possible number of arrangements is $19!$.\\
			The number of ways to select the couple which would not sit together is $\comb{10}{1}$.
			The number of ways to arrange the remaining $9$ couples is $2^9 8!$.\\
			The number of positions available for the first member of the selected couple, between two of the other couples, is $9$.
			The number of positions available for the second member of the selected couple, between two of the other couples, is $8$.
			Therefore, the total number of arrangements is $10 \cdot 2^9 8! \cdot 9 \cdot 8$.
			Therefore, the probability is $\frac{10 \cdot 2^9 8! \cdot 9 \cdot 8}{19!}$.
	\end{enumerate}
\end{solution}

\begin{question}
	In a factory, each worker was randomly allocated a phone extension with $5$ digits.
	The first digit in a manager's number is $0$, and in a non-manager worker is different than $0$.
	Find the probabilities of the following events for a manager, and for a non-manager worker.
	\begin{enumerate}
		\item $E_1$: The number has a digit which repeats.
		\item $E_2$: The digit $3$ appears in the number.
		\item $E_3$: The number is divisible by $5$.
		\item $E_4$: The number is a palindrome.
	\end{enumerate}
\end{question}

\begin{solution}
	\begin{enumerate}[leftmargin=*]
		\item
			\begin{enumerate}
				\item
					For a manager, the total possible number of extensions is $10^4$.\\
					The number of extensions with no repeating digits is $\perm{9}{4}$.\\
					Therefore, the probability of a manager's extension having no repeating digits is $\frac{\perm{9}{4}}{10^4}$.\\
					Therefore, the probability of a manager's extension having at least one repeating digit is $1 - \frac{\perm{9}{4}}{10^4}$.
				\item
					For a worker, the total possible number of extensions is $9 \cdot 10^4$.\\
					The number of extensions with no repeating digits is $\comb{9}{1} \perm{9}{4}$.\\
					Therefore, the probability of a worker's extension having no repeating digits is $\frac{\comb{9}{1} \perm{9}{4}}{9 \cdot 10^4}$.\\
					Therefore, the probability of a worker's extension having at least one repeating digit is $1 - \frac{\comb{9}{1} \perm{9}{4}}{9 \cdot 10^4}$.
			\end{enumerate}
		\item
			\begin{enumerate}
				\item
					For a manager, the total possible number of extensions is $10^4$.\\
					The number of possible positions for the digit $3$ are $4$.\\
					The number of options for the remaining digits is $10^3$.\\
					Therefore, the total number of extensions with at least one $3$ is $4 \cdot 10^3$.\\
					Therefore, the probability of a manager's extension having $3$ is $\frac{4 \cdot 10^3}{10^4}$.
				\item
					For a worker, the total possible number of extensions is $9 \cdot 10^4$.\\
					The number of possible positions for the digit $3$ are $4$.\\
					The number of options for the remaining digits is $10^3$.\\
					Therefore, the total number of extensions with at least one $3$ is $9 \cdot 4 \cdot 10^3$.\\
					Therefore, the probability of a worker's extension having $3$ is $\frac{9 \cdot 4 \cdot 10^3}{9 \cdot 10^4}$.
			\end{enumerate}
		\item
			\begin{enumerate}
				\item
					For a manager, the total possible number of extensions is $10^4$.\\
					For the extension number to be divisible by $5$, the number of options for the last digit are $2$.\\
					The number of options for the remaining digits is $10^3$.\\
					Therefore, the total number of extension numbers divisible by $5$ is $2 \cdot 10^3$.\\
					Therefore, the probability of a manager's extension number being divisible is $\frac{2 \cdot 10^3}{10^4}$.
				\item
					For a worker, the total possible number of extensions is $9 \cdot 10^4$.\\
					For the extension number to be divisible by $5$, the number of options for the last digit are $2$.\\
					The number of options for the remaining digits is $10^3$.\\
					Therefore, the total number of extension numbers divisible by $5$ is $9 \cdot 2 \cdot 10^3$.\\
					Therefore, the probability of a worker's extension number being divisible is $\frac{9 \cdot 2 \cdot 10^3}{9 \cdot 10^4}$.
			\end{enumerate}
		\item
			\begin{enumerate}
				\item
					For a manager, the total possible number of extensions is $10^4$.\\
					For the extension number to be a palindrome, the fourth digit and the fifth digit must be the same as the second digit and the first digit respectively.
					Therefore, the extension is determined by the first three digits only.\\
					For a manager, the number of options for the first digit is $1$.\\
					The number of options for the second digit is $10$.\\
					The number of options for the third digit is $10$.\\
					Therefore, the number of possible extensions is $1 \cdot 10 \cdot 10$.\\
					Therefore, the probability is $\frac{1 \cdot 10 \cdot 10}{10^4}$.
				\item
					For a worker, the total possible number of extensions is $9 \cdot 10^4$.\\
					For the extension number to be a palindrome, the fourth digit and the fifth digit must be the same as the second digit and the first digit respectively.
					Therefore, the extension is determined by the first three digits only.\\
					For a manager, the number of options for the first digit is $9$.\\
					The number of options for the second digit is $10$.\\
					The number of options for the third digit is $10$.\\
					Therefore, the number of possible extensions is $9 \cdot 10 \cdot 10$.\\
					Therefore, the probability is $\frac{9 \cdot 10 \cdot 10}{9 \cdot 10^4}$.
			\end{enumerate}
	\end{enumerate}
\end{solution}

\end{document}
