\documentclass[fleqn, a4paper, 11pt, oneside]{amsart}
\usepackage{exsheets}
\usepackage{amsmath, amssymb, amsthm} %standard AMS packages
\usepackage{marginnote} %marginnotes
\usepackage{gensymb} %miscellaneous symbols
\usepackage{commath} %differential symbols
\usepackage{xcolor} %colours
\usepackage{cancel} %cancelling terms
\usepackage[free-standing-units, space-before-unit]{siunitx} %formatting units
\usepackage{tikz, pgfplots} %diagrams
	\usetikzlibrary{calc, hobby, patterns, intersections, decorations.markings}
\usepackage{graphicx} %inserting graphics
\usepackage{hyperref} %hyperlinks
\usepackage{datetime} %date and time
\usepackage{enumerate,enumitem} %numbered lists
\usepackage{float} %inserting floats
\usepackage{circuitikz}[american voltages, american currents] %circuit diagrams
\usepackage{algpseudocode} %algorithms
\usepackage{algorithm} %algorithms
\usepackage{booktabs}
\usepackage{xspace}
\usepackage{todonotes}

\newcommand\numberthis{\addtocounter{equation}{1}\tag{\theequation}} %adds numbers to specific equations in non-numbered list of equations

\theoremstyle{definition}
\newtheorem{example}{Example}
\newtheorem{definition}{Definition}

\theoremstyle{theorem}
\newtheorem{theorem}{Theorem}

\makeatletter
\@addtoreset{section}{part} %resets section numbers in new part
\makeatother

\DeclareMathOperator{\comp}{\textsc{c}}

\DeclareMathOperator{\prob}{\mathrm{P}}

\DeclareMathOperator{\expct}{\mathrm{E}}

\DeclareMathOperator{\var}{\mathrm{V}}

\DeclareMathOperator{\pdf}{\mathrm{f}}

\DeclareMathOperator{\bin}{\mathrm{Bin}}

\DeclareMathOperator{\poi}{\mathrm{Poi}}

\DeclareMathOperator{\geo}{\mathrm{Geo}}

\DeclareMathOperator{\nb}{\mathrm{NB}}

\DeclareMathOperator{\hg}{\mathrm{HG}}

\newcommand*{\perm}[2]{{}^{#1}\mathrm{P}_{#2}}%
\newcommand*{\comb}[2]{{}^{#1}\mathrm{C}_{#2}}%

\newcommand*{\cdf}[1]{\mathrm{F}_{#1}}

\newcommand{\A}{\text{Alice}\xspace}
\newcommand{\B}{\text{Bob}\xspace}
\newcommand{\C}{\text{Charlie}\xspace}
\newcommand{\D}{\text{David}\xspace}
\newcommand{\E}{\text{Eve}\xspace}
\newcommand{\F}{\text{Frank}\xspace}

\SetupExSheets{solution/print = true}

%opening
\title
[
	Introduction to Probability and Statistics : Assignment 6
]
{
	Introduction to Probability and Statistics\\
	Assignment 6
}
\author
{
	Aakash Jog\\
	ID : 989323563
}
\date{\formatdate{13}{3}{2016}}

\begin{document}

\maketitle
%\setlength{\mathindent}{0pt}

\begin{question}
	On average, $5.2$ hurricanes hit a certain region in a year. What is the probability that there will be $3$ or fewer hurricanes hitting this year?
\end{question}

\begin{solution}
	Let $X$ be the number of hurricanes hitting the region in a year.
	Therefore,
	\begin{align*}
		X &\sim \poi(5.2)
	\end{align*}
	Therefore,
	\begin{align*}
		\prob(X \le 3) &= \sum\limits_{i = 0}^{3} \frac{e^{-\lambda} \lambda^i}{i!}\\
		&= \sum\limits_{i = 0}^{3} \frac{e^{-5.2} (5.2)^i}{i!}
	\end{align*}
\end{solution}

\begin{question}
	The number of eggs laid on a tree leaf by an insect of a certain type is a Poisson random variable with parameter $\lambda$.
	However, such a random variable can be observed only if it is positive, since if it is zero, then we cannot know that such an insect was on the leaf.
	Let $Y$ denote the observed number of eggs, then
	\begin{align*}
		\prob(Y = i) &= \prob(X = i | X > 0)
	\end{align*}
	where $X$ is a Poisson random variable with parameter $\lambda$.
	Find $\expct[Y]$.
\end{question}

\begin{solution}
	\begin{align*}
		\expct[Y] &= \sum\limits_{i = 0}^{\infty} i \prob(X = i|X > 0)\\
		&= \sum\limits_{i = 0}^{\infty} i \frac{\prob(X = i \cap X > 0)}{\prob(X > 0)}\\
		&= \sum\limits_{i = 1}^{\infty} i \frac{\prob(X = i)}{\prob(X > 0)}\\
		&= \frac{\expct[X]}{\prob(X > 0)}\\
		&= \frac{\lambda}{1 - e^{-\lambda}}
	\end{align*}
\end{solution}

\begin{question}
	A professor with ADHD never ends his classes on time.
	Let $T$ be the time in minutes that the class continues beyond its scheduled end.
	The probability distribution function of $T$ is
	\begin{align*}
		\pdf(t) &=
			\begin{cases}
				c t (5 - t)^2 &;\quad 0 \le t \le 5\\
				0 &;\quad \text{otherwise}\\
			\end{cases}
	\end{align*}
	\begin{enumerate}
		\item
			Find $c$.
		\item
			Find the cumulative distribution function of $T$.
		\item
			A student bets with her friends that in $12$ out of $13$ independent classes, the class will continue at least $2$ minutes after its scheduled end.
			What are her chances to win the bet?
	\end{enumerate}
\end{question}

\begin{solution}
	\begin{enumerate}[leftmargin=*]
		\item
			\begin{align*}
				1 &= \int\limits_{-\infty}^{\infty} \pdf(t) \dif t\\
				&= \int\limits_{0}^{5} c t (5 - t)^2 \dif t\\
				&= \frac{625 c}{12}
			\end{align*}
			Therefore,
			\begin{align*}
				c &= \frac{12}{625}
			\end{align*}
		\item
			\begin{align*}
				\cdf{T}(t) &= \int\limits_{-\infty}^{t} \pdf(s) \dif s\\
				&=
					\begin{cases}
						0 &;\quad t < 0\\
						\frac{12}{625} \left( \frac{25 t^2}{2} - \frac{10 t^3}{3} + \frac{t^4}{4} \right) &;\quad 0 \le t \le 5\\
						1 &;\quad 5 < t\\
					\end{cases}
			\end{align*}
		\item
			\begin{align*}
				\prob(T \ge 2) &= \int\limits_{0}^{2} \pdf(t) \dif t\\
				&= \int\limits_{0}^{2} \frac{12}{625} t (5 - t)^2 \dif t\\
				&= \frac{328}{625}
			\end{align*}
	\end{enumerate}
\end{solution}

\begin{question}
	$X$, the percentage of correct answers of a student in a test is distributed according to
	\begin{align*}
		\pdf(t) &=
			\begin{cases}
				c t (100 - t) &;\quad 0 \le t \le 100\\
				0 &;\quad \text{otherwise}\\
			\end{cases}
	\end{align*}
	\begin{enumerate}
		\item
			What is the value of $c$, given that $\pdf(t)$ is a legitimate probability distribution function?
		\item
			Find $\cdf{X}(t)$.
		\item
			Calculate the probability that a student will fail, i.e. get less than $55$.
		\item
			Calculate the probability of failure in the test if
			\begin{enumerate}
				\item The student's grade is higher than $54$.
				\item The student's grade is higher than $40$.
			\end{enumerate}
	\end{enumerate}
\end{question}

\begin{solution}
	\begin{enumerate}[leftmargin=*]
		\item
			\begin{align*}
				1 &= \int\limits_{-\infty}^{\infty} \pdf(t) \dif t\\
				&= \int\limits_{0}^{100} c t (100 - t)\\
				&= \left. -c \left( -50 t^2 + \frac{t^3}{3} \right) \right|_{0}^{100}\\
				&= \frac{500000 c}{3}
			\end{align*}
			Therefore,
			\begin{align*}
				c &= \frac{3}{500000}
			\end{align*}
		\item
			\begin{align*}
				\cdf{X}(x) &= \int\limits_{-\infty}^{x} \cdf{X}(x) \dif x\\
				&=
					\begin{cases}
						0 &\quad x < 0\\
						-\frac{3}{500000} \left( -50 x^2 + \frac{x^3}{3} \right) &;\quad 0 \le t \le 100\\
						1 &\quad 100 < x\\
					\end{cases}
			\end{align*}
		\item
			\begin{align*}
				\prob(X < 55) &= \int\limits_{0}^{55} \pdf(t) \dif t\\
				&= \int\limits_{0}^{55} \frac{3}{500000} t (100 - t) \dif t\\
				&= \frac{2299}{4000}
			\end{align*}
		\item
			\begin{align*}
				\prob(X < 55 | X > 54) &= \frac{\prob(54 < X < 55)}{\prob(X > 54)}\\
				&= \frac{\frac{7439}{500000}}{\frac{6877}{15625}}\\
				&= \frac{7439}{220064}
			\end{align*}
			\begin{align*}
				\prob(X < 55 | X > 40) &= \frac{\prob(40 < X < 55)}{\prob(X > 54)}\\
				&= \frac{\frac{891}{4000}}{\frac{81}{125}}\\
				&= \frac{11}{32}
			\end{align*}
	\end{enumerate}
\end{solution}

\begin{question}
	The number of minutes of playing time of a certain high school basketball player in a randomly chosen game is a random variable whose probability density function is given.
	\begin{figure}[H]
		\centering
		\begin{tikzpicture}[xscale = 0.2, yscale = 40]
			\def\xMIN{0};
			\def\xMAX{50};
			\def\yMIN{0};
			\def\yMAX{0.1};

			\begin{scope}[-stealth]
				\draw (\xMIN,0) -- (\xMAX,0) node [right] {$t$};
				\draw (0,\yMIN) -- (0,\yMAX) node [above] {$\pdf(T)$};
			\end{scope}

			\begin{scope}
				\draw (10,0) -- (10,0.025) -- (20,0.025) -- (20,0.05) -- (30,0.05) -- (30,0.025) -- (40,0.025) -- (40,0);
			\end{scope}

			\begin{scope}
				\node [left] at (0,0.025) {$0.025$};
				\node [left] at (0,0.05) {$0.05$};
				\node [below] at (10,0) {$10$};
				\node [below] at (20,0) {$20$};
				\node [below] at (30,0) {$30$};
				\node [below] at (40,0) {$40$};
			\end{scope}
		\end{tikzpicture}
	\end{figure}
	Find the probability that the player plays
	\begin{enumerate}
		\item over $15$ minutes.
		\item between $20$ and $35$ minutes.
		\item less than $30$ minutes.
		\item more than $36$ minutes.
	\end{enumerate}
\end{question}

\begin{solution}
	\begin{enumerate}[leftmargin=*]
		\item
			\begin{align*}
				\prob(T > 15) &= \int\limits_{15}^{\infty} \pdf(T) \dif t\\
				&= 0.875
			\end{align*}
		\item
			\begin{align*}
				\prob(20 < T < 35) &= \int\limits_{20}^{35} \pdf(T) \dif t\\
				&= 0.625
			\end{align*}
		\item
			\begin{align*}
				\prob(T < 30) &= \int\limits_{30}^{\infty} \pdf(T) \dif t\\
				&= 0.75
			\end{align*}
		\item
			\begin{align*}
				\prob(T > 36) &= \int\limits_{36}^{\infty} \pdf(T) \dif t\\
				&= 0.1
			\end{align*}
	\end{enumerate}
\end{solution}

\begin{question}
	For some constant $c$, the random variable $X$ has the following probability density function.
	\begin{align*}
		\pdf(x) &=
			\begin{cases}
				a x + b x^2 &;\quad 0 < x < 1\\
				0 &;\quad \text{otherwise}\\
			\end{cases}
	\end{align*}
	If
	\begin{align*}
		\expct[X] &= 0.6
	\end{align*}
	find
	\begin{enumerate}
		\item $\prob(X < 12)$.
		\item $\var(X)$.
	\end{enumerate}
\end{question}

\begin{question}
	Your company must make a scaled bid for a construction project.
	If you succeed in winning the contract by having the lowest bid, then you plan to pay another firm $\$100,000$ to do the work.
	If you believe that the minimum bid of the other participating companies can be modelled as the value of a random variable that is uniformly distributed on $(70,140)$, how much should you bid to maximize your profit?
\end{question}

\begin{question}
	At a certain bank, the amount of time that a customer spends being served by a teller is an exponential random variable with mean of $5$ minutes.
	If there is a customer in service when you enter the bank, what is the probability that he will still be with the teller after an additional $4$ minutes?
\end{question}

\end{document}
