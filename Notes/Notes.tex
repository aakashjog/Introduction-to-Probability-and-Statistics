\documentclass[titlepage, fleqn, a4paper, 12pt, twoside]{article}
\usepackage{geometry}
\usepackage{exsheets} %question and solution environments
\usepackage{amsmath, amssymb, amsthm} %standard AMS packages
\usepackage{esint} %integral signs
\usepackage{marginnote} %marginnotes
\usepackage{gensymb} %miscellaneous symbols
\usepackage{commath} %differential symbols
\usepackage{xcolor} %colours
\usepackage{cancel} %cancelling terms
\usepackage[free-standing-units]{siunitx} %formatting units
\usepackage{tikz, pgfplots} %diagrams
	\usetikzlibrary{calc, hobby, patterns, intersections, angles, quotes, spy}
\usepackage{graphicx} %inserting graphics
\usepackage{epstopdf} %converting and inserting eps graphics
\usepackage{hyperref} %hyperlinks
\usepackage{datetime} %date and time
\usepackage{enumerate, enumitem} %numbered lists
\usepackage{float} %inserting floats
\usepackage{microtype} %micro-typography
\usepackage{todonotes}
\usepackage{booktabs}

\newcommand\numberthis{\addtocounter{equation}{1}\tag{\theequation}} %adds numbers to specific equations in non-numbered list of equations

\theoremstyle{definition}
\newtheorem{example}{Example}
\newtheorem{definition}{Definition}

\theoremstyle{theorem}
\newtheorem{theorem}{Theorem}
\newtheorem{law}{Law}
\newtheorem{axiom}{Axiom}

\makeatletter
\@addtoreset{section}{part} %resets section numbers in new part
\makeatother

\newcommand\blfootnote[1]{%
	\begingroup
	\renewcommand\thefootnote{}\footnote{#1}%
	\addtocounter{footnote}{-1}%
	\endgroup
}

\renewcommand{\marginfont}{\scriptsize \color{blue}}

\renewcommand{\tilde}{\widetilde}

\DeclareMathOperator{\comp}{\textsc{c}}

\DeclareMathOperator{\prob}{\mathrm{P}}

\newcommand*{\perm}[2]{{}^{#1}\mathrm{P}_{#2}}%
\newcommand*{\comb}[2]{{}^{#1}\mathrm{C}_{#2}}%

\SetupExSheets{solution/print = true} %prints all solutions by default

%opening
\title{Introduction to Probability and Statistics}
\author{Aakash Jog}
\date{2015-16}

\begin{document}

\pagenumbering{roman}
\begin{titlepage}
\newgeometry{margin=0cm}
\maketitle
\end{titlepage}
\restoregeometry
%\setlength{\mathindent}{0pt}

\blfootnote
{	
	\begin{figure}[H]
		\includegraphics[height = 12pt]{cc.pdf}
		\includegraphics[height = 12pt]{by.pdf}
		\includegraphics[height = 12pt]{nc.pdf}
		\includegraphics[height = 12pt]{sa.pdf}
	\end{figure}
	This work is licensed under the Creative Commons Attribution-NonCommercial-ShareAlike 4.0 International License. To view a copy of this license, visit \url{http://creativecommons.org/licenses/by-nc-sa/4.0/}.
} %CC-BY-NC-SA license

\tableofcontents

\clearpage
\section{Lecturer Information}

\textbf{Dr. Galit Ashkenazi-Golan}\\
~\\
E-mail: \href{mailto:galit.ashkenazi@gmail.com}{galit.ashkenazi@gmail.com}\\

\section{Recommended Reading}

\begin{enumerate}
	\item Sheldon M. Ross: A First Course in Probability Pearson Prenticce Hall, 8th Edition, 2010.
	\item Bertsekas, Dimitri P. and Tsitsikis, John N., Introduction to Probability. Athena Science, 2nd editions, 2008.
	\item Montgomery, D.C and Runger, G.C. and Hubele, N.F. Engineering Statistics. Wiley \& Sons, NY, 4th Edition, 2007.
\end{enumerate}

\clearpage
\pagenumbering{arabic}

\part{Basics of Probability}

\section{Terminology}

\begin{definition}[Experiment]
	A situation with uncertain results is called an experiment.
\end{definition}

\begin{definition}[Sample space]
	The set of all possible outcomes of an experiment is called the sample space.
	It is denoted by $\Omega$ or $S$.
\end{definition}

\begin{definition}[Event]
	Any subset $A$ of the sample space is called an event.
\end{definition}

\begin{definition}[Intersection of sets]
	Let $A$ and $B$ be two events of sample space $\Omega$.
	The set of all outcomes that are both in $A$ and $B$ is called the intersection of $A$ and $B$.
	It is denoted by $A \cap B$.
\end{definition}

\begin{definition}[Union of sets]
	Let $A$ and $B$ be two events of sample space $\Omega$.
	The set of all outcomes that are in either of $A$ and $B$ is called the union of $A$ and $B$.
	It is denoted by $A \cup B$.
\end{definition}

\begin{definition}[Complement of set]
	Let $A$ be an event of sample space $\Omega$.
	The set of all outcomes that are not in $A$, but are in $\Omega$ is called the complement of $A$.
	It is denoted by $\overline{A}$ or $A^{\comp}$.
\end{definition}

\begin{definition}[Mutually exclusive events]
	Two events $A$ and $B$ are said to be mutually exclusive, if
	\begin{align*}
		A \cap B & = \emptyset
	\end{align*}
\end{definition}

\section{Basic Laws}

\begin{law}[Commutative Laws]
	\begin{align*}
		A \cup B & = B \cup A \\
		A \cap B & = B \cap A
	\end{align*}
\end{law}

\begin{law}[Associative Laws]
	\begin{align*}
		(A \cup B) \cup C & = A \cup (B \cup C) \\
		(A \cap B) \cap C & = A \cap (B \cap C)
	\end{align*}
\end{law}

\begin{law}[Distributive Laws]
	\begin{align*}
		(A \cup B) \cap C & = (A \cap C) \cup (B \cap C) \\
		(A \cap B) \cup C & = (A \cup C) \cap (B \cup C)
	\end{align*}
\end{law}

\begin{law}[De Morgan's Laws]
	\begin{align*}
		\overline{A_1 \cup \dots \cup A_n} & = \overline{A_1} \cap \dots \cap \overline{A_n} \\
		\overline{A_1 \cap \dots \cap A_n} & = \overline{A_1} \cup \dots \cup \overline{A_n}
	\end{align*}
\end{law}

\begin{proof}
	\begin{gather*}
		\omega \in \overline{A_a \cup \dots \cup A_n}\\
		\iff \omega \notin A_ \cup \dots \cup A_n\\
		\iff \omega \notin A_1 \text{ and } \dots \text{ and } \omega \notin A_n\\
		\iff \omega \in \overline{A_1} \text{ and } \dots \text{ and } \omega \in \overline{A_n}\\
		\iff \omega \in \overline{A_1} \cap \dots \cap \overline{A_n}
	\end{gather*}
	Similarly,
	\begin{gather*}
		\omega \in \overline{A_a \cap \dots \cap A_n}\\
		\iff \omega \notin A_ \cap \dots \cap A_n\\
		\iff \omega \notin A_1 \text{ or } \dots \text{ or } \omega \notin A_n\\
		\iff \omega \in \overline{A_1} \text{ or } \dots \text{ or } \omega \in \overline{A_n}\\
		\iff \omega \in \overline{A_1} \cup \dots \cup \overline{A_n}
	\end{gather*}
\end{proof}

\section{Axioms of Probability}

\begin{definition}[Probability]
	The probability of an event $E$ is defined to be a function which satisfies the three basic axioms.
	It is denoted by $P(E)$.
	\begin{axiom}
		\begin{equation*}
			0 \le \prob(E) \le 1
		\end{equation*}
	\end{axiom}
	
	\begin{axiom}
		\begin{align*}
			\prob(\Omega) & = 1
		\end{align*}
	\end{axiom}
	
	\begin{axiom}
		For any sequence of mutually exclusive events $A_1$, $\dots$,
		\begin{align*}
			\prob\left( \bigcup\limits_{i = 1}^{\infty} A_i \right) & = \sum\limits_{i = 1}^{\infty} \prob(A_i)
		\end{align*}
	\end{axiom}
\end{definition}

\begin{theorem}
	\begin{align*}
		\prob(\emptyset) & = 0
	\end{align*}
\end{theorem}

\begin{theorem}
	For a finite collection of mutually exclusive event $A_1$, $\dots$, $A_n$,
	\begin{align*}
		\prob\left( \bigcup\limits_{i = 1}^{n} A_i \right) & = \sum\limits_{i = 1}^{n} \prob(A_i)
	\end{align*}
\end{theorem}

\begin{theorem}
	\begin{align*}
		\prob\left( \overline{A} \right) & = 1 - \prob(A)
	\end{align*}
\end{theorem}

\begin{proof}
	\begin{align*}
		A \cap \overline{A} & = \emptyset
	\end{align*}
	Therefore, $A$ and $\overline{A}$ are mutually exclusive.
	Therefore,
	\begin{align*}
		\prob(A) + \prob\left( \overline{A} \right) & = \prob\left( A \cup \overline{A} \right) \\
                                                            & = \prob(\Omega)                           \\
                                                            & = 1                                       \\
		\therefore \prob\left( \overline{A} \right) & = 1 - \prob(A)
	\end{align*}
\end{proof}

\begin{theorem}
	\begin{align*}
		\prob(A \cup B) & = \prob(A) + \prob(B) - \prob(A \cap B)
	\end{align*}
\end{theorem}

\begin{definition}[Symmetric sample spaces]
	A sample space is said to be symmetric if the probabilities of all $\omega \in \Omega$ are the same.
\end{definition}

\section{Basics of Combinatorics}

\begin{theorem}
	The number of combinations of $k$ objects out of $n$, without repetition is
	\begin{align*}
		\binom{n}{k} & = \comb{n}{k} \\
                             & = \frac{n!}{(n - k)! k!}
	\end{align*}
\end{theorem}

\begin{theorem}
	\begin{align*}
		\perm{n}{k} & = k! \comb{n}{k} \\
                            & = \frac{n!}{(n - k)!}
	\end{align*}
	The number of permutations of $k$ objects out of $n$, without repetition is
\end{theorem}

\begin{question}
	$8$ books are to be arranged on $2$ shelves, of capacities $3$ and $5$ respectively.
	Out of the $8$ books, $2$ books are special.
	Find the probability that the two special books end up on the same shelf.
\end{question}

\begin{solution}
	Let the special books be placed first.\\
	If the first special book is placed on the longer shelf, then it has $5$ available positions, and the second special book has $4$ available positions.\\
	If the first special book is placed on the shorter shelf, then it has $3$ available positions, and the second special book has $2$ available positions.\\
	In either case, the number of ways of arranging the remaining $6$ books in the remaining positions is $6!$.\\
	Therefore, the total number of arrangements satisfying the conditions is $(5 \cdot 4 \cdot 6! + 3 \cdot 2 \cdot 6!)$.
	The total number of arrangements is $8!$.
	Therefore, the required probability is $\frac{5 \cdot 4 \cdot 6! + 3 \cdot 2 \cdot 6!}{8!}$.
\end{solution}

\begin{solution}
	Let the places for the special books be allotted first.\\
	If the first special book is assigned a place on the longer shelf, then it has $5$ available positions, and the second special book has $4$ available positions.\\
	If the first special book is assigned a place on the shorter shelf, then it has $3$ available positions, and the second special book has $2$ available positions.\\
	The total number of arrangements is $8!$.
	Therefore, the required probability is $\frac{5 \cdot 4 + 3 \cdot 2}{8 \cdot 7}$.
\end{solution}

\begin{solution}
	If the special books are to be placed on the longer shelf, the possible combinations are $\binom{5}{2}$.\\
	If the special books are to be placed on the shorter shelf, the possible combinations are $\binom{3}{2}$.\\
	To total possible arrangements are $\binom{8}{2}$.\\
	Therefore, the required probability is $\frac{\binom{5}{2} + \binom{3}{2}}{\binom{8}{2}}$.
\end{solution}

\begin{question}
	Ben is going to celebrate the beginning of the year of the dragon.
	He lives close to two pubs.
	The probability that he would go to pub A is $0.5$ and the probability that the would go to pub B is $0.4$.
	In addition, the probability that he would go to at least one of the two venues is $0.8$.
	\begin{enumerate}
		\item What is the sample space?
		\item What is the probability that he would go to both pubs?
		\item What is the probability that he would go to exactly one pub?
	\end{enumerate}
\end{question}

\begin{solution}
	\begin{enumerate}[leftmargin=*]
		\item
			Let $A$ be the event that he would go to pub $A$, and let $B$ be the event that he goes to pub $B$.
			Therefore,
			\begin{align*}
				S & = \left\{ A \cap B^{\comp} , A^{\comp} \cap B , A \cap B , A^{\comp} \cap B^{\comp} \right\}
			\end{align*}
		\item
			The probability that he would go to both pubs is
			\begin{align*}
				\prob(A \cap B) & = \prob(A) + \prob(B) - \prob(A \cup B) \\
                                                & = 0.5 + 0.4 - 0.8                       \\
                                                & = 0.1
			\end{align*}
		\item
			The probability that he would go to exactly one pub is
			\begin{align*}
				\prob\left( (A \cup B) \setminus (A \cap B) \right) & = \prob(A \cup B) - \prob(A \cap B) \\
                                                                                    & = 0.8 - 0.1                         \\
                                                                                    & = 0.7
			\end{align*}
	\end{enumerate}
\end{solution}

\begin{question}
	A lady crosses three traffic signals, with red and green lights only, on the way to her dog's hairdresser.\\
	The probabilities of encountering $0$, $1$, and $2$ red lights are $0.4$, $0.1$, $0.2$ respectively.\\
	Find the probabilities of
	\begin{enumerate}
		\item Encountering at least one red light.
		\item Encountering at least one green light.
		\item Encountering an odd number of red lights.
	\end{enumerate}
\end{question}

\begin{solution}
	\begin{enumerate}[leftmargin=*]
		\item
			The required probability is
			\begin{align*}
				\prob(\text{$1$ red}) + \prob(\text{$2$ red}) + \prob(\text{$3$ red}) & = 1 - \prob(\text{$0$ red}) \\
                                                                                                      & = 1 - 0.4                   \\
                                                                                                      & = 0.6
			\end{align*}
		\item
			The required probability is
			\begin{align*}
				\prob(\text{$1$ green}) + \prob(\text{$2$ green}) + \prob(\text{$3$ green}) & = \prob(\text{$0$ red}) + \prob(\text{$1$ red}) + \prob(\text{$2$ red}) \\
                                                                                                            & = 0.4 + 0.1 + 0.2                                                       \\
                                                                                                            & = 0.7
			\end{align*}
		\item
			The required probability is
			\begin{align*}
				\prob(\text{$1$ red}) + \prob(\text{$3$ red}) & = 1 - \prob(\text{$0$ red}) - \prob(\text{$2$ red}) \\
                                                                              & = 1 - 0.4 - 0.2                                     \\
                                                                              & = 0.4
			\end{align*}
	\end{enumerate}
\end{solution}

\end{document}
