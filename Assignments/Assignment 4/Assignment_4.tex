\documentclass[fleqn, a4paper, 11pt, oneside]{amsart}
\usepackage{exsheets}
\usepackage{amsmath, amssymb, amsthm} %standard AMS packages
\usepackage{marginnote} %marginnotes
\usepackage{gensymb} %miscellaneous symbols
\usepackage{commath} %differential symbols
\usepackage{xcolor} %colours
\usepackage{cancel} %cancelling terms
\usepackage[free-standing-units, space-before-unit]{siunitx} %formatting units
\usepackage{tikz, pgfplots} %diagrams
	\usetikzlibrary{calc, hobby, patterns, intersections, decorations.markings}
\usepackage{graphicx} %inserting graphics
\usepackage{hyperref} %hyperlinks
\usepackage{datetime} %date and time
\usepackage{enumerate,enumitem} %numbered lists
\usepackage{float} %inserting floats
\usepackage{circuitikz}[american voltages, american currents] %circuit diagrams
\usepackage{algpseudocode} %algorithms
\usepackage{algorithm} %algorithms
\usepackage{booktabs}
\usepackage{xspace}

\newcommand\numberthis{\addtocounter{equation}{1}\tag{\theequation}} %adds numbers to specific equations in non-numbered list of equations

\theoremstyle{definition}
\newtheorem{example}{Example}
\newtheorem{definition}{Definition}

\theoremstyle{theorem}
\newtheorem{theorem}{Theorem}

\makeatletter
\@addtoreset{section}{part} %resets section numbers in new part
\makeatother

\DeclareMathOperator{\comp}{\textsc{c}}

\DeclareMathOperator{\prob}{\mathrm{P}}

\DeclareMathOperator{\expct}{\mathrm{E}}

\DeclareMathOperator{\var}{\mathrm{V}}

\newcommand*{\perm}[2]{{}^{#1}\mathrm{P}_{#2}}%
\newcommand*{\comb}[2]{{}^{#1}\mathrm{C}_{#2}}%

\newcommand{\A}{\text{Alice}\xspace}
\newcommand{\B}{\text{Bob}\xspace}
\newcommand{\C}{\text{Charlie}\xspace}
\newcommand{\D}{\text{David}\xspace}
\newcommand{\E}{\text{Eve}\xspace}
\newcommand{\F}{\text{Frank}\xspace}

\SetupExSheets{solution/print = true}

%opening
\title
[
	Introduction to Probability and Statistics : Assignment 4
]
{
	Introduction to Probability and Statistics\\
	Assignment 4
}
\author
{
	Aakash Jog\\
	ID : 989323563
}
\date{\formatdate{30}{3}{2016}}

\begin{document}

\maketitle
%\setlength{\mathindent}{0pt}

\begin{question}
	Suppose that a random variable $X$ is equal to the number of hits obtained by a certain baseball player in his next $3$ bats.
	If
	\begin{align*}
		\prob(X = 1) & = 0.3 \\
		\prob(X = 2) & = 0.2 \\
		\prob(X = 0) & = 3 \prob(X = 3)
	\end{align*}
	find $\expct[X]$.
\end{question}

\begin{solution}
	\begin{align*}
		\sum\limits_{i = 0}^{3} \prob(X = i)  & = 1     \\
		\therefore 0.3 + 0.2 + 4 \prob(X = 3) & = 1     \\
		\therefore 4 \prob(X = 3)             & = 0.5   \\
		\therefore \prob(X = 3)               & = 0.125 \\
		\therefore \prob(X = 0)               & = 0.375
	\end{align*}
	Therefore,
	\begin{align*}
		\expct[E] & = \sum\limits_{i = 0}^{3} i \prob(X = i)            \\
                          & = (0) (0.375) + (1) (0.3) + (2) (0.2) + (3) (0.125) \\
                          & = 0 + 0.3 + 0.4 + 0.375                             \\
                          & = 1.075
	\end{align*}
\end{solution}

\begin{question}
	Suppose that $X$ takes on one of the values $0$, $1$, or $2$.
	If for some constant $c$,
	\begin{align*}
		\prob(X = i) & = c \prob(X = i - 1)
	\end{align*}
	for $i = 1$ and $i = 2$, find $\expct[X]$ in terms of $c$.
\end{question}

\begin{solution}
	\begin{align*}
		\sum\limits_{i = 0}^{2} \prob(X = i)                        & = 1 \\
		\therefore \prob(X = 0) + c \prob(X = 0) + c^2 \prob(X = 0) & = 1 \\
		\therefore \prob(X = 0)                                     & = \frac{1}{1 + c + c^2}
	\end{align*}
	Therefore,
	\begin{align*}
		\expct[E] & = \sum\limits_{i = 0}^{2} i \prob(X = i)                                                                                           \\
                          & = (0) \left( \frac{1}{1 + c + c^2} \right) + (1) \left( \frac{c}{1 + c + c^2} \right) + (2) \left( \frac{c^2}{1 + c + c^2} \right) \\
                          & = \frac{c + 2 c^2}{1 + c + c^2}
	\end{align*}
\end{solution}

\begin{question}
	Suppose that
	\begin{align*}
		\prob(X = 0) & = 1 - \prob(X = 1)
	\end{align*}
	If
	\begin{align*}
		\expct[X] & = 3 \var(X)
	\end{align*}
	find $\prob(X = 0)$.
\end{question}

\begin{solution}
	\begin{align*}
		\expct[X] & = 3 \var(X)
	\end{align*}
	Therefore $X$ is a Bernoulli random variable.\\
	Therefore, let the probability
	\begin{align*}
		\prob(X = 0) & = 1 - p \\
		\prob(X = 1) & = p
	\end{align*}
	Therefore,
	\begin{align*}
		\expct[X]        & = 3 \var(X)   \\
		\therefore p     & = 3 p (1 - p) \\
		\therefore p     & = 3 p - 3 p^2 \\
		\therefore 3 p^2 & = 2 p         \\
	\end{align*}
	Therefore, either $p = 0$ or $p = \frac{2}{3}$.
	Therefore, $\prob(X = 0)$ is either $1$ or $\frac{1}{3}$.
\end{solution}

\begin{question}
	There are two coins in a bin.
	When one of them is flipped, it lands on `Heads' with probability $0.6$, and when the other is flipped, it lands on `Heads' with probability $0.3$.\\
	One of these coins is to be randomly chosen and then flipped.
	Without knowing which coin is chosen, you can bet any amount up to $\$10$, and you then either win that amount if the coin comes up `Heads' or lose it if it comes up `Tails'.\\
	Suppose, however, that an insider is willing to sell you, for an amount $C$, the information as to which coin was selected.
	What is your expected payoff if you buy this information?
	Note that if you but it and then bet $x$, you will end up wither winning $x - C$ or $-x - C$.
	Also, for what values of $C$ does it pay to purchase the information?
\end{question}

\begin{solution}
	Let the amount bet be $x$.
	Therefore, without knowing which coin is selected, the probability of the outcome being `Heads' is $(0.5) (0.6) + (0.5) (0.3)$, i.e. $0.45$.\\
	Let the winnings be denoted by the random variable $X$.
	Therefore the expected winnings are
	\begin{align*}
		E[X] & = (x) (0.45) + (-x) (0.55) \\
                     & = -0.1 x
	\end{align*}
	Therefore, as the expected winnings are negative, one would never bet anything, i.e. bet $0$.\\
	If the information is bought from the insider, one would bet the maximum allowed money if the coin with $0.6$ chance of `Heads' is selected, and no money if the other coin is selected.
	Therefore, the expected winnings are
	\begin{align*}
		E[X] & = (0.6) (10 - C) + (0.4) (-10 - C) + (0.3) (0 - C) + (0.7) (0 - C) \\
                     & = 6 - 0.6 C -4 - 0.4 C - C                                         \\
                     & = 2 - 2 C
	\end{align*}
	Therefore, it pays to buy the information if and only if
	\begin{align*}
		2 - 2 C & > 0 \\
		\iff C  & < 1
	\end{align*}
\end{solution}

\begin{question}
	A philanthropist writes a positive number $x$ on a piece of red paper, shows the paper to an impartial observer, and then turns it face down on a table.
	The observer then flips a fair coin.
	If it shows he writes the value $2 x$ if it shows `Heads', and the value $\frac{x}{2}$ if it shows `Tails', on a piece of blue paper, which he then turns face down on the table.
	Without knowing either the value $x$ or the result of the coin flip, you have to option of turning over either the red or the blue piece of paper.
	After doing so and observing the number written on that paper, you may elect to receive as a reward either that amount or the unknown amount written on the other piece of paper.\\
	Suppose that you would like your expected reward to be large, argue that there is no reason to turn over the red paper first, because if you do so, then no matter what value you observe, it is always better to switch to the blue paper.
\end{question}

\begin{solution}
	The probability of $x$ being written on the red paper is $1$.\\
	The probability of $2 x$ being written on the blue paper is $0.5$, and that of $\frac{x}{2}$ being written on the blue paper is $0.5$.\\
	Therefore, if the red paper is turned over first, the expected value of the number written is $x$.\\
	If the blue paper is turned over first, the expected value of the number written is $x + \frac{x}{4}$.\\
	Therefore, as the expected value of the number written on the blue paper is always higher than that written on the red paper, there is no point in turning over the red paper first.
	Even if one does turn over the red paper first, it is better to then switch to the blue paper.
\end{solution}

\end{document}
