\documentclass[fleqn, a4paper, 11pt, oneside]{amsart}
\usepackage{exsheets}
\usepackage{amsmath, amssymb, amsthm} %standard AMS packages
\usepackage{marginnote} %marginnotes
\usepackage{gensymb} %miscellaneous symbols
\usepackage{commath} %differential symbols
\usepackage{xcolor} %colours
\usepackage{cancel} %cancelling terms
\usepackage[free-standing-units, space-before-unit]{siunitx} %formatting units
\usepackage{tikz, pgfplots} %diagrams
	\usetikzlibrary{calc, hobby, patterns, intersections, decorations.markings}
\usepackage{graphicx} %inserting graphics
\usepackage{hyperref} %hyperlinks
\usepackage{datetime} %date and time
\usepackage{enumerate,enumitem} %numbered lists
\usepackage{float} %inserting floats
\usepackage{circuitikz}[american voltages, american currents] %circuit diagrams
\usepackage{algpseudocode} %algorithms
\usepackage{algorithm} %algorithms
\usepackage{booktabs}
\usepackage{xspace}

\newcommand\numberthis{\addtocounter{equation}{1}\tag{\theequation}} %adds numbers to specific equations in non-numbered list of equations

\theoremstyle{definition}
\newtheorem{example}{Example}
\newtheorem{definition}{Definition}

\theoremstyle{theorem}
\newtheorem{theorem}{Theorem}

\makeatletter
\@addtoreset{section}{part} %resets section numbers in new part
\makeatother

\DeclareMathOperator{\comp}{\textsc{c}}

\DeclareMathOperator{\prob}{\mathrm{P}}

\newcommand*{\perm}[2]{{}^{#1}\mathrm{P}_{#2}}%
\newcommand*{\comb}[2]{{}^{#1}\mathrm{C}_{#2}}%

\newcommand{\A}{\text{Alice}\xspace}
\newcommand{\B}{\text{Bob}\xspace}
\newcommand{\C}{\text{Charlie}\xspace}
\newcommand{\D}{\text{David}\xspace}
\newcommand{\E}{\text{Eve}\xspace}
\newcommand{\F}{\text{Frank}\xspace}

\SetupExSheets{solution/print = true}

%opening
\title
[
	Introduction to Probability and Statistics : Assignment 3
]
{
	Introduction to Probability and Statistics\\
	Assignment 3
}
\author
{
	Aakash Jog\\
	ID : 989323563
}
\date{\formatdate{23}{3}{2016}}

\begin{document}

\maketitle
%\setlength{\mathindent}{0pt}

\begin{question}
	You ask your neighbour to water a sickly plant while you are on vacation.\\
	Without water, it will die with probability $0.8$.\\
	With water, it will die with probability $0.15$.\\
	You are $90\%$ certain that your neighbour will remember to water the plant.\\
	\begin{enumerate}
		\item What is the probability that the plant will be alive when you return?
		\item If the plant is dead upon your return, what is the probability that your neighbour forgot to water it?
	\end{enumerate}
\end{question}

\begin{solution}
	\begin{enumerate}[leftmargin=*]
		\item
			The probability of the plant being alive is $(0.9) (0.85) + (0.1) (0.2)$.\\
			Therefore, the probability of the plant being alive is $0.785$.
		\item
			The total probability that the plant will be dead is $0.215$.\\
			The probability that the plant will be dead and the neighbour watered it is $(0.9) (0.15)$.\\
			The probability that the plant will be dead and the neighbour did not water it is $(0.1) (0.8)$.
			Therefore, if the plant is dead, the probability that the neighbour did not water it is $\frac{0.08}{0.215}$.
	\end{enumerate}
\end{solution}

\begin{question}
	Six balls are to be randomly chosen from an urn containing $8$ red, $10$ green, and $12$ blue balls.
	\begin{enumerate}
		\item What is the probability that at least one red ball is chosen?
		\item Given that no red balls are chosen, what is the conditional probability that there are exactly $2$ green balls among the $6$ chosen balls?
	\end{enumerate}
\end{question}

\begin{solution}
	\begin{enumerate}
		\item
			The total number of ways to choose $6$ balls is $\comb{30}{6}$.\\
			The number of ways to choose $0$ red balls is $\comb{22}{6}$.\\
			Therefore, the probability of selecting $0$ red balls is $\frac{\comb{22}{6}}{\comb{30}{6}}$.
			Therefore, the probability of selecting at least $1$ red ball is $1 - \frac{\comb{22}{6}}{\comb{30}{6}}$.
		\item
			If $0$ red balls are chosen and exactly $2$ green balls are chosen, the remaining $4$ balls must be blue.\\
			Therefore, the number of ways of choosing $2$ green balls and $4$ blue balls is $\comb{10}{2} \comb{12}{4}$.\\
			The total number of ways to choose $6$ non-red balls is $\comb{22}{6}$.\\
			Therefore, the probability is $\frac{\comb{10}{2} \comb{12}{4}}{\comb{22}{6}}$.
	\end{enumerate}
\end{solution}

\begin{question}
	A coin having probability $0.8$ of landing on `Heads' is flipped.
	\A observes the result, and rushes off to tell \B.
	However, with probability $0.4$, \A will have forgotten the result by the time she reaches \B.
	If \A has forgotten, then rather than admitting this to \B, she is equally likely to tell \B that the coin landed on `Heads' or that it landed on `Tails'.
	If she does remember, she tells \B the correct result.
	\begin{enumerate}
		\item What is the probability that \B is told that the coin landed on heads?
		\item What is the probability that \B is told the correct result?
		\item Given that \B is told that the coin landed on heads, what is the probability that it did in fact land on `Heads'?
	\end{enumerate}
\end{question}

\begin{solution}
	\begin{enumerate}[leftmargin=*]
		\item
			If \A forgets the result, the probability that \B is told that the result is `Heads' is $0.5$.\\
			If \A remembers the result, the probability that \B is told that the result is `Heads' is $0.8$\\
			Therefore, the total probability that \B is told that the result is `Heads' is $(0.4) (0.5) + (0.6) (0.8)$.
			Therefore, the probability is $0.68$.
		\item
			The probability that \B is told the correct result if \A remembers the result is $1$.\\
			The probability that \B is told the correct result if \A forgets the result is $0.5$.\\
			Therefore, the probability that \B is told the correct result is $(0.6) (1) + (0.4) (0.5)$
			Therefore, the probability is $0.8$.
	\end{enumerate}
\end{solution}

\begin{question}
	In the urn, there are $5$ red and $4$ white balls.
	You sequentially draw $3$ balls randomly without replacement.
	What is the probability that all balls are white?
\end{question}

\begin{solution}
	The total number of ways to select $3$ balls from the urn is $\comb{9}{3}$.\\
	The number of ways to select $3$ white balls from $4$ white balls is $\comb{4}{3}$.\\
	Therefore, the probability is $\frac{\comb{4}{3}}{\comb{9}{3}}$.
\end{solution}

\begin{question}
	Let $A$ and $B$ be events having positive probability.
	State whether each of the following statements is necessarily true, necessarily false, or possibly true.
	\begin{enumerate}
		\item If $A$ and $B$ are mutually exclusive, then they are independent.
		\item If $A$ and $B$ are independent, then they are mutually exclusive.
		\item $\prob(A) = \prob(B) = 0.6$, and $A$ and $B$ are mutually exclusive.
		\item $\prob(A) = \prob(B) = 0.6$, and $A$ and $B$ are independent.
	\end{enumerate}
\end{question}

\begin{solution}
	\begin{enumerate}
		\item
			If $A$ and $B$ are mutually exclusive,
			\begin{align*}
				\prob(A \cap B) &= 0
			\end{align*}
			$A$ and $B$ can be independent if and only if
			\begin{align*}
				\prob(A \cap B) &= \prob(A) \prob(B)\\
				\iff 0 &= \prob(A) \prob(B)
			\end{align*}
			However, as $A$ and $B$ have positive probability, this statement is necessarily false.
		\item
			If $A$ and $B$ are independent,
			\begin{align*}
				\prob(A \cap B) &= \prob(A) \prob(B)\\
				\therefore \prob(A \cap B) &> 0
			\end{align*}
			$A$ and $B$ can be mutually exclusive if and only if
			\begin{align*}
				\prob(A) \prob(B) &= 0\\
				\iff \prob(A \cap B) &= 0
			\end{align*}
			However, as $A$ and $B$ have positive probability, this statement is necessarily false.
		\item
			$A$ and $B$ are mutually exclusive if and only if
			\begin{align*}
				\prob(A) + \prob(B) &\le 1\\
				\iff 0.6 + 0.6 &\le 1
			\end{align*}
			Therefore, this statement is necessarily false.
		\item
			$A$ and $B$ are independent if and only if
			\begin{align*}
				\prob(A) \prob(B) &= \prob(A \cap B)\\
				\iff (0.6) (0.6) &= \prob(A \cap B)\\
				\iff \prob(A \cap B) &= 0.36
			\end{align*}
			Therefore, this statement is possibly true.
	\end{enumerate}
\end{solution}

\begin{question}
	Arrange the following from most likely to occur to least likely to occur.
	\begin{enumerate}
		\item A fair coin lands on `Heads'.
		\item Three independent trials, each of which is a success with probability $0.8$, all result in successes.
		\item Seven independent trials, each of which is a success with probability $0.9$, all result in successes.
	\end{enumerate}
\end{question}

\begin{solution}
	The probability that a fair coin lands on `Heads' is $0.5$.\\
	The probability that three independent trials with probability of success $0.8$ all result in successes is $(0.8)^3$.
	Therefore, the probability is $0.512$.\\
	The probability that seven independent trials with probability of success $0.9$ all result in successes is $(0.9)^7$.
	Therefore, the probability is $0.4782969$.\\
	Therefore, the descending order of probabilities is
	\begin{enumerate}
		\item Three independent trials, each of which is a success with probability $0.8$, all result in successes.
		\item A fair coin lands on `Heads'.
		\item Seven independent trials, each of which is a success with probability $0.9$, all result in successes.
	\end{enumerate}
\end{solution}

\end{document}
