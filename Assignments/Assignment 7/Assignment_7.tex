\documentclass[fleqn, a4paper, 11pt, oneside]{amsart}
\usepackage{exsheets}
\usepackage{amsmath, amssymb, amsthm} %standard AMS packages
\usepackage{marginnote} %marginnotes
\usepackage{gensymb} %miscellaneous symbols
\usepackage{commath} %differential symbols
\usepackage{xcolor} %colours
\usepackage{cancel} %cancelling terms
\usepackage[free-standing-units, space-before-unit]{siunitx} %formatting units
\usepackage{tikz, pgfplots} %diagrams
	\usetikzlibrary{calc, hobby, patterns, intersections, decorations.markings}
\usepackage{graphicx} %inserting graphics
\usepackage{hyperref} %hyperlinks
\usepackage{datetime} %date and time
\usepackage{enumerate,enumitem} %numbered lists
\usepackage{float} %inserting floats
\usepackage{circuitikz}[american voltages, american currents] %circuit diagrams
\usepackage{algpseudocode} %algorithms
\usepackage{algorithm} %algorithms
\usepackage{booktabs}
\usepackage{xspace}
\usepackage{todonotes}

\newcommand\numberthis{\addtocounter{equation}{1}\tag{\theequation}} %adds numbers to specific equations in non-numbered list of equations

\theoremstyle{definition}
\newtheorem{example}{Example}
\newtheorem{definition}{Definition}

\theoremstyle{theorem}
\newtheorem{theorem}{Theorem}

\makeatletter
\@addtoreset{section}{part} %resets section numbers in new part
\makeatother

\DeclareMathOperator{\comp}{\textsc{c}}

\DeclareMathOperator{\prob}{\mathrm{P}}

\DeclareMathOperator{\expct}{\mathrm{E}}

\DeclareMathOperator{\var}{\mathrm{V}}

\DeclareMathOperator{\pdf}{\mathrm{f}}

\DeclareMathOperator{\bin}{\mathrm{Bin}}

\DeclareMathOperator{\poi}{\mathrm{Poi}}

\DeclareMathOperator{\geo}{\mathrm{Geo}}

\DeclareMathOperator{\nb}{\mathrm{NB}}

\DeclareMathOperator{\hg}{\mathrm{HG}}

\DeclareMathOperator{\uniform}{\mathrm{U}}

\newcommand*{\perm}[2]{{}^{#1}\mathrm{P}_{#2}}%
\newcommand*{\comb}[2]{{}^{#1}\mathrm{C}_{#2}}%

\newcommand*{\cdf}[1]{\mathrm{F}_{#1}}

\newcommand{\A}{\text{Alice}\xspace}
\newcommand{\B}{\text{Bob}\xspace}
\newcommand{\C}{\text{Charlie}\xspace}
\newcommand{\D}{\text{David}\xspace}
\newcommand{\E}{\text{Eve}\xspace}
\newcommand{\F}{\text{Frank}\xspace}

\SetupExSheets{solution/print = true}

%opening
\title
[
	Introduction to Probability and Statistics : Assignment 7
]
{
	Introduction to Probability and Statistics\\
	Assignment 7
}
\author
{
	Aakash Jog\\
	ID : 989323563
}
\date{\formatdate{4}{5}{2016}}

\begin{document}

\maketitle
%\setlength{\mathindent}{0pt}

\begin{question}
	The number of minutes of playing time of a certain high school basketball player in a randomly chosen game is a random variable whose probability density function is given.
	\begin{figure}[H]
		\centering
		\begin{tikzpicture}[xscale = 0.2, yscale = 40]
			\def\xMIN{0};
			\def\xMAX{50};
			\def\yMIN{0};
			\def\yMAX{0.1};

			\begin{scope}[-stealth]
				\draw (\xMIN,0) -- (\xMAX,0) node [right] {$t$};
				\draw (0,\yMIN) -- (0,\yMAX) node [above] {$\pdf(T)$};
			\end{scope}

			\begin{scope}
				\draw (10,0) -- (10,0.025) -- (20,0.025) -- (20,0.05) -- (30,0.05) -- (30,0.025) -- (40,0.025) -- (40,0);
			\end{scope}

			\begin{scope}
				\node [left] at (0,0.025) {$0.025$};
				\node [left] at (0,0.05) {$0.05$};
				\node [below] at (10,0) {$10$};
				\node [below] at (20,0) {$20$};
				\node [below] at (30,0) {$30$};
				\node [below] at (40,0) {$40$};
			\end{scope}
		\end{tikzpicture}
	\end{figure}
	Find the probability that the player plays
	\begin{enumerate}
		\item over $15$ minutes.
		\item between $20$ and $35$ minutes.
		\item less than $30$ minutes.
		\item more than $36$ minutes.
	\end{enumerate}
\end{question}

\begin{solution}
	\begin{enumerate}[leftmargin=*]
		\item
			\begin{align*}
				\prob(T > 15) &= \int\limits_{15}^{\infty} \pdf(T) \dif t\\
				&= 0.875
			\end{align*}
		\item
			\begin{align*}
				\prob(20 < T < 35) &= \int\limits_{20}^{35} \pdf(T) \dif t\\
				&= 0.625
			\end{align*}
		\item
			\begin{align*}
				\prob(T < 30) &= \int\limits_{30}^{\infty} \pdf(T) \dif t\\
				&= 0.75
			\end{align*}
		\item
			\begin{align*}
				\prob(T > 36) &= \int\limits_{36}^{\infty} \pdf(T) \dif t\\
				&= 0.1
			\end{align*}
	\end{enumerate}
\end{solution}

\begin{question}
	For some constant $c$, the random variable $X$ has probability density function
	\begin{align*}
		\pdf(x) &=
			\begin{cases}
				c x^4 &;\quad 0 < x < 2\\
				0 &;\quad \text{otherwise}\\
			\end{cases}
	\end{align*}
	Find $\expct[X]$ and $\var(X)$.
\end{question}

\begin{solution}
	As $\pdf(x)$ is a probability density function,
	\begin{align*}
		1 &= \int\limits_{-\infty}^{\infty} \pdf(x) \dif x\\
		&= \int\limits_{0}^{2} c x^4\\
		&= c \left( \frac{32}{5} \right)
	\end{align*}
	Therefore,
	\begin{align*}
		c &= \frac{5}{32}
	\end{align*}
	Therefore,
	\begin{align*}
		\expct[X] &= \int\limits_{-\infty}^{\infty} x \pdf(x) \dif x\\
		&= \frac{5}{32} \int\limits_{0}^{2} x^5 \dif x\\
		&= \frac{5}{32} \frac{64}{6}\\
		&= \frac{5}{3}
	\end{align*}
	Therefore,
	\begin{align*}
		\expct\left[ g(X) \right] &= \int\limits_{-\infty}^{\infty} g(x) \pdf(x) \dif x\\
		\therefore \expct\left[ X^2 \right] &= \frac{5}{32} \int\limits_{0}^{2} x^2 x^4\\
		&= \frac{5}{32} \int\limits_{0}^{2} x^6\\
		&= \frac{5}{32} \frac{128}{7}\\
		&= \frac{20}{7}
	\end{align*}
	Therefore,
	\begin{align*}
		\var(X) &= \expct\left[ X^2 \right] - \expct[X]^2\\
		&= \frac{20}{7} - \frac{25}{9}\\
		&= \frac{5}{63}
	\end{align*}
\end{solution}

\begin{question}
	Your company must make a scaled bid for a construction project.
	If you succeed in winning the contract by having the lowest bid, then you plan to pay another firm $\$100,000$ to do the work.
	If you believe that the minimum bid of the other participating companies can be modelled as the value of a random variable that is uniformly distributed on $(70,140)$, how much should you bid to maximize your profit?
\end{question}

\begin{question}
	At a certain bank, the amount of time that a customer spends being served by a teller is an exponential random variable with mean of $5$ minutes.
	If there is a customer in service when you enter the bank, what is the probability that he will still be with the teller after an additional $4$ minutes?
\end{question}

\begin{question}
	\A is leaving work in a uniformly distributed time between $7$ and $9$, i.e.
	\begin{align*}
		X &\sim \uniform(7,9)
	\end{align*}
	The time it takes her to commute home is
	\begin{align*}
		Y &= 1 + \frac{1}{X}
	\end{align*}
	Find the density and distribution of $Y$.
\end{question}

\begin{question}
	A randomly chosen IQ test taker obtains a score that is approximately a normal random variable with mean $100$ and standard deviation $15$.
	Find the probability that the score of such a person is
	\begin{enumerate}
		\item above $125$.
		\item between $90$ and $110$.
	\end{enumerate}
\end{question}

\begin{solution}
	\begin{enumerate}[leftmargin=*]
		\item 
			Let $X$ be the obtained score.\\
			Let
			\begin{align*}
				Z &= \frac{X - \mu}{\sigma}\\
				&= \frac{X - 100}{15}
			\end{align*}
			Therefore, $Z$ is a standard normal random variable.
			Therefore,
			\begin{align*}
				\prob(X > 125) &= \prob\left( Z > \frac{125 - 100}{15} \right)\\
				&= \prob\left( \frac{5}{3} \right)\\
				&= 1 - \Phi\left( \frac{5}{3} \right)
			\end{align*}
		\item
			Let $X$ be the obtained score.\\
			Let
			\begin{align*}
				Z &= \frac{X - \mu}{\sigma}\\
				&= \frac{X - 100}{15}
			\end{align*}
			Therefore, $Z$ is a standard normal random variable.
			Therefore,
			\begin{align*}
				\prob(90 < X < 100) &= \prob\left( \frac{90 - 100}{15} < Z < \frac{110 - 100}{15} \right)\\
				&= \prob\left( \frac{-2}{3} < Z < \frac{2}{3} \right)\\
				&= \Phi\left( \frac{2}{3} \right) - \Phi\left( -\frac{2}{3} \right)\\
				&= \Phi\left( \frac{2}{3} \right) - \left( 1 - \Phi\left( \frac{2}{3} \right) \right)\\
				&= 2 \Phi\left( \frac{2}{3} \right) - 1
			\end{align*}
	\end{enumerate}
\end{solution}

\begin{question}
	The life of a certain type of automobile tyre is normally distributed with mean $34,000$ miles and standard deviation $4,000$ miles.
	\begin{enumerate}
		\item What is the probability that such a tyre lasts over $40,000$ miles?
		\item What is the probability that such a tyre lasts between $30,000$ and $35,000$ miles?
		\item Given that such a tyre has survived $30,000$ miles, what is the conditional probability that the tyre survives another $10,000$ miles?
	\end{enumerate}
\end{question}

\begin{solution}
	\begin{enumerate}[leftmargin=*]
		\item
			Let $X$ be the life of the tyre.\\
			Let
			\begin{align*}
				Z &= \frac{X - \mu}{\sigma}\\
				&= \frac{X - 34000}{4000}
			\end{align*}
			Therefore, $Z$ is a standard normal random variable.
			Therefore,
			\begin{align*}
				\prob(X > 40000) &= \prob\left( Z > \frac{40000 - 34000}{4000} \right)\\
				&= \prob\left( Z > \frac{3}{2} \right)\\
				&= 1 - \Phi\left( \frac{3}{2} \right)
			\end{align*}
		\item
			Let $X$ be the life of the tyre.\\
			Let
			\begin{align*}
				Z &= \frac{X - \mu}{\sigma}\\
				&= \frac{X - 34000}{4000}
			\end{align*}
			Therefore, $Z$ is a standard normal random variable.
			Therefore,
			\begin{align*}
				\prob(30000 < X < 35000) &= \prob\left( \frac{30000 - 34000}{4000} < Z < \frac{35000 - 34000}{4000} \right)\\
				&= \prob\left( -1 < Z < \frac{1}{4} \right)\\
				&= \Phi\left( \frac{1}{4} \right) - \Phi(-1)\\
				&= \Phi\left( \frac{1}{4} \right) - \left( 1 - \varphi(1) \right)\\
				&= \Phi\left( \frac{1}{4} \right) + \Phi(1) - 1
			\end{align*}
		\item
			Let $X$ be the life of the tyre.\\
			Let
			\begin{align*}
				Z &= \frac{X - \mu}{\sigma}\\
				&= \frac{X - 34000}{4000}
			\end{align*}
			Therefore, $Z$ is a standard normal random variable.
			Therefore,
			\begin{align*}
				\prob(X > 30000) &= \prob\left( Z > \frac{30000 - 34000}{4000} \right)\\
				&= \prob(Z > -1)\\
				&= 1 - \Phi(1)\\
				\prob(X > 40000) &= \prob\left( Z > \frac{40000 - 34000}{4000} \right)\\
				&= \prob\left( Z > \frac{3}{2} \right)\\
				&= 1 - \Phi\left( \frac{3}{2} \right)
			\end{align*}
			Therefore,
			\begin{align*}
				\prob(X > 40000 | X > 30000) &= \frac{\prob(X > 40000 \cap X > 30000}{\prob(X > 30000)}\\
				&= \frac{\prob(X > 40000)}{\prob(X > 30000)}\\
				&= \frac{1 - \Phi\left( \frac{3}{2} \right)}{1 - \Phi(1)}
			\end{align*}
	\end{enumerate}
\end{solution}

\begin{question}
	The annual rainfall in Cleveland, Ohio, is approximately a normal random variable with mean $40.2$ inches and standard deviation $8.4$ inches.
	Let $A_i$ be the event that the rainfall in the next $i$th year exceeds $44$ inches, and fume that all $A_i$ are independent.
	Find the probability that
	\begin{enumerate}
		\item next year's rainfall will exceed $44$ inches?
		\item the yearly rainfall in exactly $3$ of the next $7$ years will exceed $44$ inches?
	\end{enumerate}
\end{question}

\begin{solution}
	\begin{enumerate}[leftmargin=*]
		\item
			Let $X$ be the rainfall in a year.\\
			Let
			\begin{align*}
				Z &= \frac{X - \mu}{\sigma}\\
				&= \frac{X - 30.2}{8.4}
			\end{align*}
			Therefore, $Z$ is a standard normal random variable.
			Therefore,
			\begin{align*}
				\prob(X > 44) &= \prob\left( Z > \frac{44 - 30.2}{8.4} \right)\\
				&= \prob\left( Z > \frac{13.8}{8.4} \right)\\
				&= \prob\left( Z > \frac{23}{14} \right)\\
				&= 1 - \Phi\left( \frac{23}{14} \right)
			\end{align*}
		\item
			For any year, as the rainfall in every year is independent of the previous years, the probability that the rainfall will exceed $44$ inches is
			\begin{align*}
				\prob(X > 44) &= 1 - \Phi\left( \frac{23}{14} \right)
			\end{align*}
			Therefore, the probability that the rainfall will exceed $44$ inches in exactly $3$ of the next $7$ years is $\frac{7}{3} \left( 1 - \Phi\left( \frac{23}{14} \right) \right)^3 \left( \Phi\left( \frac{23}{14} \right) \right)^4$.
	\end{enumerate}
\end{solution}

\end{document}
